% !begin_codebook
\end{multicols}
\includepdf[pages={-}, angle=90,pagecommand={\pagestyle{fancy}}]{codebookpart2}

\newpage 
% hexagon grid
% adapted from https://tex.stackexchange.com/questions/61429/how-to-draw-a-hexagonal-grid-with-numbers-in-the-cells
\begin{tikzpicture}[hexa/.style= {shape=regular polygon,regular polygon sides=6,minimum size=1cm, draw,inner sep=0,anchor=south,rotate=30}]
\foreach \j in {0,...,22}{%
\pgfmathsetmacro\end{29+(\j-\j/2)}
\pgfmathtruncatemacro\what{\j-\j/2}
  \foreach \i in {\what,...,\end}{%
  \node[hexa] (h\i;\j) at ({(\i-\j/2)*sin(60)},{\j*0.75}) {};}  }      
\end{tikzpicture}
\iffalse
\newpage
% large hexagon
\begin{tikzpicture}[hexa/.style= {shape=regular polygon,regular polygon sides=6,minimum size=2cm, draw,inner sep=0,anchor=south,rotate=30}]
\foreach \j in {0,...,10}{%
\pgfmathsetmacro\end{14+(\j-\j/2)}
\pgfmathtruncatemacro\what{\j-\j/2}
  \foreach \i in {\what,...,\end}{%
  \node[hexa] (h\i;\j) at ({(\i-\j/2)*sin(60)*2},{\j*0.75*2}) {};}  }      
\end{tikzpicture}
\newpage 
% large hexagon
\begin{tikzpicture}[hexa/.style= {shape=regular polygon,regular polygon sides=6,minimum size=2cm, draw,inner sep=0,anchor=south,rotate=30}]
\foreach \j in {0,...,10}{%
\pgfmathsetmacro\end{14+(\j-\j/2)}
\pgfmathtruncatemacro\what{\j-\j/2}
  \foreach \i in {\what,...,\end}{%
  \node[hexa] (h\i;\j) at ({(\i-\j/2)*sin(60)*2},{\j*0.75*2}) {};}  }      
\end{tikzpicture}
\newpage 
\mbox{}
\newpage 
\mbox{}
\newpage 
\mbox{}
\newpage 
\mbox{}
\newpage 
\mbox{}
\newpage 
\mbox{}
\newpage 
\mbox{}
\newpage 
\mbox{}
\newpage 
\mbox{}
\fi
\newpage
\pagestyle{empty} % in case you accidentially print theses pages, they will be empty and the paper can be reused
\setcounter{page}{4}
\addcontentsline{toc}{section}{Radixsort 50M 64 bit integers as single array in 1 sec}
\addcontentsline{toc}{section}{FFT 10-15M length/sec}
\addcontentsline{toc}{section}{Berlekamp-Massey $O(LN)$}
\mbox{}
\newpage 
\addcontentsline{toc}{section}{Mo's algorithm}
\addcontentsline{toc}{section}{LIS}
\mbox{}
\newpage 
\addcontentsline{toc}{section}{Persistent queue}
\addcontentsline{toc}{section}{Dynamic N-dimensional Fenwick tree}
\mbox{}
\newpage 
\addcontentsline{toc}{section}{Fenwick}
\addcontentsline{toc}{section}{Offline Li Chao tree}
\addcontentsline{toc}{section}{Wavelet matrix}
\mbox{}
\newpage 
\mbox{}
\newpage 
\addcontentsline{toc}{section}{Convex hull trick (static)}
\addcontentsline{toc}{section}{Range tree}
\addcontentsline{toc}{section}{LCA with binary lifting}
\mbox{}
\newpage 
\addcontentsline{toc}{section}{DSU with rollback}
\addcontentsline{toc}{section}{Segment tree beats}
\mbox{}
\newpage 
\addcontentsline{toc}{section}{Dominator tree}
\mbox{}
\newpage 
\addcontentsline{toc}{section}{Offline dynamic connectivity}
\mbox{}
\newpage 
\addcontentsline{toc}{section}{DFS}
\addcontentsline{toc}{section}{Mo on trees}
\addcontentsline{toc}{section}{Parallel binary search}
\addcontentsline{toc}{section}{DSU}
\mbox{}
\newpage 
\addcontentsline{toc}{section}{Heavy-light decomposition}
\addcontentsline{toc}{section}{DSU on tree sack}
\mbox{}
\newpage 
\addcontentsline{toc}{section}{Treap}
\addcontentsline{toc}{section}{Online Li chao tree}
\mbox{}
\newpage 
\addcontentsline{toc}{section}{Link-Cut tree}
\mbox{}
\newpage 
\addcontentsline{toc}{section}{Tetration}
\addcontentsline{toc}{section}{Pollard rho factorization}
\mbox{}
\newpage
\addcontentsline{toc}{section}{Python BigInt}
\addcontentsline{toc}{section}{Disjoint sparse table}
\addcontentsline{toc}{section}{SCC}
\addcontentsline{toc}{section}{Online bridges}
\mbox{}
\newpage
\addcontentsline{toc}{section}{Manhattan MST}
\mbox{}
\newpage
\addcontentsline{toc}{section}{Directed MST}
\mbox{}
\newpage
\addcontentsline{toc}{section}{Bipartite matching (Hopcroft-Karp)}
\addcontentsline{toc}{section}{Single source shortest path ($V\cdot\max w + E$)}
\addcontentsline{toc}{section}{Two edge connected components}
\mbox{}
\newpage
\addcontentsline{toc}{section}{Offline dynamic MST}
\addcontentsline{toc}{section}{Maximum independent set}
\mbox{}
\newpage
\addcontentsline{toc}{section}{Persistent union find}
\addcontentsline{toc}{section}{Suffix array}
\mbox{}
\newpage
\addcontentsline{toc}{section}{Duval}
\addcontentsline{toc}{section}{Z-algorithm}
\addcontentsline{toc}{section}{Aho-Corasick}
\addcontentsline{toc}{section}{Suffix automaton}
\mbox{}
\newpage
\addcontentsline{toc}{section}{Mo's algorithm with updates}
\mbox{}
\end{landscape}
\end{document}
% !end_codebook
